Mit Verschlüsselung sind kryptografisch-mathematische Methoden gemeint, um Nachrichten (z.B. Text) in eine neue Form zu bringen, die nicht ohne weiteres (z.B. ohne Schlüssel) in die ursprüngliche Form gebracht werden kann.

Bei der Betrachtung der Verschlüsselung von E-Mail Verkehr muss erst definiert werden welche Teile des Verkehrs verschlüsselt werden. Hier gibt es mehrere Schichten. Die erste Schicht sind die Inhaltsdaten der E-Mail, gewissermaßen der Body und eventuell auch die Header. Die zweite Schicht ist die Sitzung zwischen E-Mail Server und Client. Dies gilt für alle involvierten Protokolle SMTP, POP3 und IMAP4, die diese Verschlüsselung unterstützen müssen, um die einzelnen Kommandos vom Client zum Server und die Antworten vom Server zum Client über einen kryptografischen Kanal zu transferieren.

Zunächst werden Technologien bezüglich der Verschlüsselung des E-Mail Bodies betrachtet und nachfolgend Technologien auf Protokoll-Ebene untersucht. Dabei werden auch Probleme dieser Verfahren und Methoden aufgezeigt.
