Wie bereits erwähnt bezeichnet Anonymität im Kontext dieser Arbeit das Verschleiern von Metadaten. Nicht immer ist die Trennung von Inhaltsdaten und Metadaten streng nachvollziehbar. So ist es beispielsweise diskutierbar, ob die allgemeinen Informationen einer SMTP Sitzung ohne Betrachtung der E-Mail Nachricht selbst als Metadaten gelten. Als Metadaten im Kontext dieses Kapitels sind aber vor allem Verbindungsdaten wie IP-Adressen gemeint.

Das SMTP Protokoll sowie die Protokolle POP3 und IMAP liefern über ihre Spezifikation keinerlei Methoden eine solche Anonymisierung durchzuführen. Ebenso sind keine Protokoll-Erweiterungen bekannt, die dieses leisten.

Allerdings sind protokollunabhängige Lösungen möglich. Eine davon ist die MUA-Erweiterung \QuoteM{TorBirdy}.
