TLS bezeichnet das Protokoll \QuoteM{Transport Layer Security}, welches im RFC 5246 \RefIt{rfc5246} definiert ist. Es ist ein Protokoll auf Applikationsebene, welches den Datenschutz und die Datenintegrität zweier kommunizierender Applikationen sicherstellen soll. Dabei besteht es aus zwei Schichten, einerseits aus dem \QuoteM{TLS Record Protocol} und dem \QuoteM{TLS Handshake Protocol}. \QuoteIndirect{rfc5246}{S. 4}

Das \QuoteM{TLS Record Protocol} stellt dabei die unterste Schicht dar, direkt über dem TCP Protokoll und stellt sicher, dass die Verbindung selbst verschlüsselt ist. Dies findet über ein symmetrisches Verschlüsselungsverfahren statt. Dabei ist der gemeinsame Schlüssel auf die Laufzeit einer Sitzung begrenzt. Ferner definiert dieses Protokoll wie die Integrität von Nachrichten über Hashfunktionen wie SHA-1 sichergestellt wird. \QuoteIndirect{rfc5246}{S. 4}

Das \QuoteM{TLS Handshake Protocol} hingegen ist im \QuoteM{TLS Record Protocol} eingebettet und übernimmt die Funktion der Authentifizierung. Auf dieser Ebene fällt ebenso die Entscheidung, welcher Verschlüsselungsalgorithmus benutzt wird und der Austausch der kryptografischen Schlüssel. Erst nach diesem Prozess wird die eigentliche Sitzung aktiv und Sitzungsdaten können gesendet werden. Dieser Prozess findet über asymmetrische Verschlüsselung basierend auf dem Public-Key Verfahren statt. Darauf basierend ist dann der Austausch des symmetrischen Sitzungsschlüssels sicher. \QuoteIndirect{rfc5246}{S. 4}

Der öffentliche Schlüssel des Servers wird in der Praxis häufig von einem weiteren öffentlichen Schlüssel einer sogenannten \QuoteM{Certificate Authority} signiert. Die verbindende Gegenstelle muss also nur dem Schlüssel der Certificate Authority vertrauen. Applikationen wie Webbrowser liefern häufig ein Paket von vertrauten Schlüsseln solcher Certificate Authorities mit. Öffentliche Schlüssel können allerdings auch \QuoteM{self-signed} sein ohne Verbindung zu einer Certificate Authority. Diese müssen dann allerdings beim Empfänger manuell verifiziert und zugelassen werden.
\QuoteIndirect{Ferguson:2003:PC:862106}{Kapitel 19 bis 21}

Die Schwächen und Probleme des TLS Protokolls und des darüber liegenden Zertifikats-Systems im einzelnen zu analysieren würde den Rahmen dieser Arbeit sprengen. Einige der bekannten Angriffe auf das Protokoll wurden in RFC 7457 \RefIt{rfc7457} veröffentlicht. Für einige davon existieren Lösungen, entweder durch Konfigurationsänderungen am Server-System oder durch Erweiterungen der TLS Spezifikation.
Festzuhalten ist aber lediglich, dass TLS das wohl am weitesten verbreitete Verfahren ist, um die Kommunikation im WWW, aber auch die Kommunikation von und zu E-Mail Servern zu verschlüsseln und das Problem eines \QuoteM{Man-in-the-Middle-Angriff}, kurz MITMA \RefIt{mitm}, zu lösen.

Die Protokolle SMTPS, POP3S und IMAPS sind also nur Bezeichner für die Protokolle SMTP, POP3 und IMAP, die auf den dedizierten Ports 465, 995 und 993 ansprechbar sind und nur über das TLS Protokoll ansprechbar sind. Für diese existiert keine konkrete Spezifikation.

