\begin{abstract}
\thispagestyle{empty}

Im Angesicht globaler Überwachung durch Regierungen, %toref
schwer kontrollierbarer Sammlung und Speicherung von Benutzer-Daten durch Anbieter von Internetdiensten %toref
und fehlender Aufklärung über Konsequenzen internetbasierter Kommunikation soll diese Arbeit sowohl einen Überblick über aktuelle Probleme von E-Mail-Kommunikation, die die Privatsphäre des Benutzers betreffen, als auch eine Untersuchung möglicher Lösungen leisten.

Die Vielzahl möglicher Kommunikationsmittel im Internet erlaubt uns nicht, dieses Thema abstrakt zu behandeln. Oft unterscheiden sich diese fundamental: in der Art der Benutzung, dem verwendeten Protokoll, den zugrunde liegenden kryptografischen Algorithmen, den aktuellen Implementierungen und inhärenten Problemen.

Deshalb konzentriert sich diese Arbeit auf eines der ältesten digitalen Nachrichten- Übertragungsverfahren, welches schon im Arpanet %toref
erste Anwendung fand: die E-Mail. Dabei wird die E-Mail sowohl als Technologie aber auch als Kommunikationsform betrachtet und ihre Bedeutung aufgezeigt.

Der Überblick über aktuelle Probleme wird alle Aspekte behandeln, die relevant für die Privatsphäre des Benutzers sind. Dies beinhaltet konkrete Daten der Kommunikation wie Inhalt, aber auch Metadaten. %toexp: Meta-Daten

Bei der Betrachtung möglicher Lösungen werden zunächst bereits existierende aufgeführt, seien sie experimentell, schon implementiert oder nur konzeptionell. Nachfolgend wird ein alternativer Vorschlag inklusive Konzeption, rudimentärer Protokollbeschreibung, beinhaltender Algorithmen und Proof of Concept Implementierung der wissenschaftlichen Gemeinschaft zur Falsifikation unterbreitet.

Dieses Konzept soll als Idee, weniger als vollständige Lösung angesehen werden. Viele daraus folgende Probleme werden nicht ausreichend gelöst werden können. Diese werden jedoch nachfolgend untersucht und Vorschläge zur Weiterentwicklung unterbreitet.

Abschließend wird diskutiert, ob das vorgeschlagene Konzept ausreichend Potenzial hat, um bei entsprechender Weiterentwicklung eine angemessene Lösung sein zu können.

\end{abstract}