Ein E-Mail System ist nur nutzbar und praktikabel, wenn es einen hohen Grad an Zuverlässigkeit hat. Sowohl die Probleme der Fehlerbehandlung als auch die Verschlüsselung, das Routing und diverse Probleme bei der Spam-Abwehr können die Zuverlässigkeit des Systems beeinträchtigen.

Deshalb muss eine Protokollerweiterung bzw. ein Mechanismus entwickelt werden, der es erlaubt, ohne Verlust von Anonymität, das Erreichen einer E-Mail beim Empfänger bestätigen zu lassen. Denkbar wäre nach einem bestimmten Intervall auf eine Bestätigungsmail, welche über eine neue zufällige Route versendet wird, zu warten. Hier stellt sich allerdings die Frage, ob über die Intervalle statistische Analysen möglich sind. Ebenso kann die Bestätigungsmail fehlschlagen und nicht ankommen.