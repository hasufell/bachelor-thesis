Neben der Wahl des Verschlüsselungsalgorithmus ist im selben Kontext das Problem eines Public-Key Systems äußerst relevant.

Dies beinhaltet mehrere Punkte. Zum einen muss noch spezifiziert werden, welches Schlüsselsystem (ein Paar aus öffentlichem und privatem Schlüssel) überhaupt verwendet wird. Weiterhin muss festgelegt werden, in welcher Weise die öffentlichen Schlüssel verbreitet werden. Dies könnte beispielsweise ähnlich wie bei DKIM \RefIt{rfc6376} erfolgen, welches die öffentlichen Schlüssel im DNS System hinterlegt \QuoteIndirect{rfc6376}{S. 6}. Diese können dann automatisch abgerufen werden. Zusätzlich kann DNSSEC \RefIt{rfc4035} verwendet werden, um das System insgesamt robuster zu machen. Denkbar ist allerdings auch ein Schlüsselsystem mit einer zentralen Autorität.

