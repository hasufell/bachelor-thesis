Nun müssen die Implementierungen noch adäquat in das Plugin-System integriert werden. Haraka Plugins werden erst ausgeführt, wenn sie in der Konfigurationsdatei \verb#config/plugins# in der korrekten Reihenfolge aufgelistet sind. Ebenso müssen einige weitere Anpassungen gemacht werden, damit ein Testlauf möglich wird. Beispielsweise muss der \verb#get_mx# Hook überschrieben werden, da hier nur lokale Systeme ohne MX-Record vorliegen. Die Implementierung ist ebenso im Anhang einsehbar. % toref

Die Konfigurationsdatei für die Hops ist folgendermaßen definiert:
\begin{mail}{Hop-Konfiguration}{hopconfig}
my_mx
access
dnsbl
helo.checks
open_relay
data.headers
enable_parse_body
my_transaction
max_unrecognized_commands   
\end{mail}

Die Konfigurationsdatei für das SMTP-Gateway ist folgendermaßen definiert:
\begin{mail}{MSA-Konfiguration}{msaconfig}
my_mx
access
dnsbl
helo.checks
mail_from.is_resolvable
open_relay
data.headers
enable_parse_body
create_myst_mail
max_unrecognized_commands    
\end{mail}

Der Hauptunterschied liegt lediglich in Zeile 8 und Zeile 9, welche die Plugins beschreiben, die über \verb#data_post# das Transaktions-Objekt verändern.
