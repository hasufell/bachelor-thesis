Dieses Unterkapitel behandelt die weitverbreiteten E-Mail Protokolle SMTP, POP3 und IMAP4, welche die Basis heutiger internetbasierter E-Mail Kommunikation bilden. Für diese Protokolle gibt es eine Vielzahl von Erweiterungen, die allerdings immer auf den bestehenden Protokollen aufbauen und diese selbst nicht verändern, damit die gemeinsame Schnittstelle nicht verloren geht.

Während das SMTP Protokoll die Kommunikation zwischen E-Mail Servern definiert, behandeln die Protokolle POP3 und IMAP4 die Verwaltung und den Zugriff auf E-Mails durch den Benutzer, nicht deren Transport zwischen Servern. Obwohl IMAP4 das modernere System ist,
verwenden viele Server noch POP3,
weshalb es hier der Vollständigkeit halber mit aufgeführt wird.