Als sehr genau spezifizierte Alternative zur E-Mail muss Dark Mail, bzw. \QuoteM{Dark Internet Mail Environment} \RefIt{darkmail} genannt werden. Dark Mail verwirft jegliche bekannten Protokolle wie SMTP und entwickelt ein gänzlich neues E-Mail Konzept mit den Protokollen \QuoteDirect{Dark Mail Transfer Protocol}{darkmail}{S. 82-87} und \QuoteDirect{Dark Mail Access Protocol}{darkmail}{S. 108}. Ebenso kommt ein eigenes Datenformat namens \QuoteDirect{D/MIME}{darkmail}{S. 68-81} zum Einsatz.

Auf die genauen Einzelheiten kann hier nicht eingegangen werden, da Dark Mail sehr komplex ist. Wichtig zu bemerken ist allerdings, dass es eine dem Onion Routing \RefIt{onionrouting1} \RefIt{onionrouting2} ähnliche Methodik benutzt. Ebenso kommt verschachtelte Verschlüsselung und ein durchdachtes Schlüsselsystem zum Einsatz. Allerdings ist Dark Mail auf Implementierungsebene noch nicht fertig und es existiert zum derzeitigen Zeitpunkt kein Release \RefIt{libdime}.
