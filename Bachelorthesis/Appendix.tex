\begin{appendix}

\chapter{Beiliegende Datenträger}
Dieser Arbeit liegt eine CD bei. Nachfolgend wird der Inhalt aufgelistet und beschrieben.

\section{Inhaltsliste}

\begin{itemize}
	\setlength\itemsep{\Itemizespace}
	\item CD 1
	\begin{itemize}
		\setlength\itemsep{\Itemizespace}
		\item Thesis
			\begin{itemize}
				\setlength\itemsep{\Itemizespace}
				\item Binary
				\item Source
			\end{itemize}
		\item Webseiten
		\item RFCs in Textform
		\item Implementierung
			\begin{itemize}
				\setlength\itemsep{\Itemizespace}
				\item Dockerfiles
				\item Source
				\item Konfigurationsdateien
			\end{itemize}
		\item Dokumentation
			\begin{itemize}
				\item Handbuch
			\end{itemize}
	\end{itemize}
\end{itemize}

\clearpage
\section{Inhaltsbeschreibung}
Die Implementierung befindet sich im Unterordner \verb#Implementierung# und beinhaltet dort auch den Quellcode für die Docker Images \verb#alpine-haraka#, \verb#alpine-haraka-gateway# und \verb#alpine-swaks#. Die Haraka Plugins sind also jeweils in den Ordnern \\ \verb#Implementierung/alpine-haraka/plugins# und \\ \verb#Implementierung/alpine-haraka-gateway/plugins#.

Der Latex Quellcode befindet sich im Unterordner \verb#LaTeX# und ein Symlink zur PDF existiert im Root-Ordner als \verb#Bachelorthesis.pdf#.

Alle RFCs sind in Textform im Unterordner \verb#RFCs# aufzufinden. Ebenso sind alle referenzierten Webseiten statisch im Unterordner \verb#Webseiten# verfügbar.

\end{appendix}

